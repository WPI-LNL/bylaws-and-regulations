\chapter{Rain Policy}

\section{Introduction}

``Inclement weather'' is a generic term often used to describe weather
conditions that are either unsafe or undesirable for outdoor events. Inclement
weather can come in many different forms, some of which are outlined below. This
policy is designed to safeguard event participants, staff, and equipment in the
event of inclement weather. In general, if weather conditions could cause danger
or unnecessary discomfort, an outdoor event should be postponed or cancelled.

\section{Weather Plans}
All outdoor events are required to have a documented weather plan. This plan
should include who from the client can make a weather call and what to do in
case of inclement weather.

If the plan involves moving the event to another date and/or an indoor location,
both the date and/or location must be fully confirmed in 25live by the Events
Office.

\section{Weather Calls}
Lens and Lights will cancel or postpone any outdoor event if weather conditions
present a danger to staff, participants, or equipment. A weather call may be
made by the LNL Technical Director (or designee), LNL Vice President (or
designee), LNL Crew Chiefs, or the designated Client Representative.

\section{Fees}
If a weather call is made resulting in event cancellation, fees will apply as
documented in the LNL Cancellation Policy.

\section{Postponing or Location Changes}
If a weather call is made resulting in a scope change, whether due to a changed
date or location, the LNL Vice President will adjust the quote to reflect the
new scope of the event.

\section{Example Weather Events}
The following is a non-exhaustive list of weather events that may result in
weather calls being made:
\begin{enumerate}
    \item Rain 
    \item Extreme heat
    \item Thunderstorms
    \item High wind
    \item Snow
\end{enumerate}
